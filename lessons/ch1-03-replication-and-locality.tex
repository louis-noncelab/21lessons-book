\chapter{복제와 지역성}
\label{les:3}

\begin{chapquote}{루이스 캐롤, \textit{이상한 나라의 앨리스}}
토끼의 화난 목소리가 들려왔다 \enquote{팻, 팻! 어디야?}
\end{chapquote}

%Quantum mechanics aside, locality is a non-issue in the physical world.
%The question \textit{\enquote{Where is X?}} can be answered in a meaningful way, no
%matter if X is a person or an object. In the digital world, the question
%of \textit{where} is already a tricky one, but not impossible to answer. Where
%are your emails, really? A bad answer would be \enquote{the cloud}, which is
%just someone else's computer. Still, if you wanted to track down every
%storage device which has your emails on it you could, in theory, locate
%them.
복잡한 양자역학은 제쳐두고, 지역성은 물리 세계에서 문제가 되지 않는다. \enquote{X가 어디에 있지?}라는 
질문은 X가 사람이건 물건이건 정확히 대답할 수 있다. 하지만 디지털 세계에서 이 질문은 
조금 까다롭긴 하지만 대답은 가능하다. 이메일이 어디에 있냐는 질문에 성의 없는 대답은 \enquote{클라우드에 있어} 일 것이다.
그래도 이론적으로 모든 저장장치를 추적하면 이메일의 위치를 찾을 수 있을 것이다.

%With bitcoin, the question of \enquote{where} is \textit{really} tricky. Where,
%exactly, are your bitcoins?
비트코인의 경우 \enquote{비트코인은 어디에 있어?}라는 질문은 정말 까다롭다. 비트코인은 정말 어디에 있는 걸까?

\begin{quotation}\begin{samepage}
\enquote{나는 수술 후 눈을 뜨고 주의를 둘러보며 수 없이 물었다. `여기가 어디야?'}
\begin{flushright} -- 다니엘 데넷\footnote{Daniel Dennett, \textit{Where Am I?}~\cite{where-am-i}}
\end{flushright}\end{samepage}\end{quotation}

%The problem is twofold: First, the distributed ledger is distributed by
%full replication, meaning the ledger is everywhere. Second, there are no
%bitcoins. Not only physically, but \textit{technically}.
문제는 두 가지다. 첫째, 분산 원장은 모든 기록을 복제하여 어디에나 있다. 둘째, 비트코인은 없다. 
물리적일 뿐만 아니라 기술적으로도 없다.

%Bitcoin keeps track of a set of unspent transaction outputs, without
%ever having to refer to an entity which represents a bitcoin. The
%existence of a bitcoin is inferred by looking at the set of unspent
%transaction outputs and calling every entry with 100 million base
%units a bitcoin.
비트코인은 미지불 트랜잭션 출력(UTXO)의 집합에 기록된다.
모든 미지불 트랜잭션 출력을 탐색하여 1억을 단위\footnote{역자: 1BTC = 1sats}로 기록되는 모든 데이터를 호출하면 비트코인의 존재를 추론할 수 있다.

\begin{quotation}\begin{samepage}
\enquote{그것은 지금 어디에 있는가? 이 순간 어느 곳을 지나는 중인가?[...] 무엇보다 비트코인은
없다. 정말로 없다. 존재하지 않는다. 장부 항목이 공유된 장부상에 존재한다. [...] 
비트코인은 물리적 위치에 존재하지 않는다. 본질적으로 장부는 모든 물리적 위치에 존재한다.
지리학의 상식은 통하지 않는다. 당신의 상식은 여기서 통하지 않는다.}
\begin{flushright} -- 피터 반 발켄버그\footnote{Peter Van Valkenburgh on the What Bitcoin Did podcast, episode 49 \cite{wbd049}}
\end{flushright}\end{samepage}\end{quotation}

%So, what do you actually own when you say \textit{\enquote{I have a bitcoin}} if
%there are no bitcoins? Well, remember all these strange words which you were
%forced to write down by the wallet you used? Turns out these magic words are
%what you own: a magic spell\footnote{The Magic Dust of Cryptography: How digital
%information is changing our society \cite{gigi:magic-spell}} which can be used
%to add some entries to the public ledger --- the keys to \enquote{move} some bitcoins.
%This is why, for all intents and purposes, your private keys \textit{are} your
%bitcoins. If you think I'm making all of this up feel free to send me your
%private keys.
비트코인이 존재하지 않는다면 \enquote{난 비트코인을 갖고 있다.}라고 말할 때, 당신이 실제
소유하고 있는 것이 무엇일까? 당신이 사용했던 지갑에 의해 강제로 적어 두어야 했던 이상한 단어들을 기억하는가?
이 마법의 단어\footnote{The Magic Dust of Cryptography: How digital
information is changing our society \cite{gigi:magic-spell}}는 당신이 공개된 장부에 항목을 추가하는 데
사용할 수 있는, 즉 비트코인을 전송할 때 필요한 열쇠이다. 그렇기 때문에 당신의 개인키가 자체가 비트코인이다. 나의 
말이 거짓 같다면, 당신의 개인키를 나에게 보내라. 

\paragraph{비트코인은 지역성이 얼마나 까다로운 비즈니스인지 나에게 알려주었다.}

% ---
%
% #### Through the Looking-Glass
%
% - [The Magic Dust of Cryptography: How digital information is changing our society][a magic spell]
%
% #### Down the Rabbit Hole
%
% - [Where Am I?][Daniel Dennett] by Daniel Dennett
% - 🎧 [Peter Van Valkenburg on Preserving the Freedom to Innovate with Public Blockchains][wbd049] WBD #49 hosted by Peter McCormack
%
% <!-- Through the Looking-Glass -->
% [a magic spell]: 
%
% <!-- Down the Rabbit Hole -->
% [Daniel Dennett]: https://www.lehigh.edu/~mhb0/Dennett-WhereAmI.pdf
% [1st Amendment]: https://en.wikipedia.org/wiki/First_Amendment_to_the_United_States_Constitution
% [wbd049]: https://www.whatbitcoindid.com/podcast/coin-centers-peter-van-valkenburg-on-preserving-the-freedom-to-innovate-with-public-blockchains
%
% <!-- Wikipedia -->
% [alice]: https://en.wikipedia.org/wiki/Alice%27s_Adventures_in_Wonderland
% [carroll]: https://en.wikipedia.org/wiki/Lewis_Carroll
