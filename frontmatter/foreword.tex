\chapter*{추천글}
\pdfbookmark{Foreword}{foreword}

%Some call it a religious experience. Others call it Bitcoin.
누군가는 이것을 종교적 경험이라고 이야기한다. 
다른 사람들은 이것을 비트코인이라 부른다.

%I first met Gigi in one of my spiritual homes -- Riga, Latvia -- the home of
%\textit{The Baltic Honeybadger} Conference, where the most fervent of the
%Bitcoin faithful make a yearly pilgrimage. After a deep lunchtime conversation,
%the bond Gigi and I forged was as set in stone as a Bitcoin transaction that was
%processed when we first shook hands a few hours prior.

열렬한 비트코인 지지자들이 매년 방문하는 라트비아 리가의 한 컨퍼런스에서 Gigi를 처음 만났다. 
긴 점심 후 Gigi와 나눈 악수는 비트코인의 트랜잭션처럼 확고했다.

%My other spiritual home, Christ Church, Oxford, where I had the privilege to
%study for my MBA, was where I had my \enquote{Rabbit Hole} moment. Like Gigi, I
%transcended the economic, technical and social realms, and was spiritually
%enveloped by Bitcoin. After \enquote{buying high} in the November 2013 bubble,
%there were several extremely hard-learned lessons to be had in the relentlessly
%crushing and seemingly never-ending 3-year bear market. These 21 Lessons would
%indeed have served me very well in that time. Many of these lessons are simply
%natural truths that, to the uninitiated, are obscured by an opaque, fragile
%film. By the end of this book however, the fa\c{c}ade will fragment fiercely.

나의 영적인 고향이자 MBA를 공부했던 옥스퍼드 크라이스트 처치는 \enquote{토끼 굴}같은 순간을 보낸 곳이다.
나는 Gigi처럼 경제, 기술, 사회적 영역에서 초월하여 정신적 영역으로 비트코인을 바라보고 있었다. 
2013년 11월 버블에서 \enquote{고점 구매}를 하고 나서 극도로 힘들었던 약세장을 견디며 극단적이고 확고한 교훈을 얻었다.
이 스물한 가지의 교훈은 나에게 큰 도움이 되었다. 
내가 깨우친 대다수의 교훈은 색안경을 끼고 있는 초보자들에게 간단하고 자연적인 진리로 받아들여지게 될 것이다. 
이 책이 끝날 무렵에는 당신의 상식은 산산조각이 날 것이다.

%On a crystal-clear night in Oxford in late-August 2016, just a few weeks after
%the knife twisted in my heart again when the Bitfinex Exchange was hacked, I sat
%in quiet contemplation at Christ Church’s Master’s Garden. Times were tough, and
%I was at my mental and emotional breaking-point after what seemed to be a
%lifetime of torture; not because of financial loss, but of the crushing
%spiritual loss I felt being isolated in my world view. If only there were
%resources like this one at the time to see that I was not alone. The Master’s
%Garden is a very special place to me and many who came before me over the
%centuries. It was there where one Charles Dodgson, a Math Tutor at Christ
%Church, observed one of his young pupils, Alice Liddell, the daughter of the
%Dean of Christ Church. Dodgson, better known by his pen-name, Lewis Carroll,
%used Alice and The Garden as his inspiration, and in the magic of that hallowed
%turf, I stared deeply into the crypto-chasm, and it stared blazingly back,
%annihilating my arrogance, and slapping my self-pride square in the face. I was
%finally at peace.

비트파이넥스의 해킹 소식으로 심장을 후벼파는 아픔을 겪은 몇 주 뒤인 2016년 늦은 8월 청명한 밤에 나는
크라이스트 처치의 마스터스 가든에서 사색을 즐기고 있었다. 
나는 힘든 시간을 보내며 정신적, 정서적 한계점에 이르렀다.
금전적 손실보다는 영적 손실 때문에 나는 나의 세계관에서 고립되었다는 느낌을 받았다. 
내가 혼자가 아니라는 것을 느낄 수 있었으면 참 좋았을 것이다. 
마스터스 정원은 수 세기에 걸쳐 많은 사람에게 특별한 장소이다.
이 정원은 크라이스트 처치의 수학교사인 찰스 도슨이 그의 제자 중 한 명이자 학장의 딸인 앨리스 리델을 지켜봤던 의미 있는 장소이다.
루이스 캐롤\footnote{역자: 이상한 나라 앨리스의 저자}이라는 필명으로 알려진 도슨은 앨리스와 이 정원에서 영감을 얻었고, 
나는 정원의 그 신성한 잔디의 마법 속에서 암호의 틈(Crypto-chasm)을 발견하였다.
나는 그것을 응시하자 나의 오만함이 소멸하는 것을 느꼈다. 
그리고 나는 마침내 자존심을 내려놓았고 평안함을 얻을 수 있었다. 

%21 Lessons takes you on a true Bitcoin journey; not just a journey of
%philosophy, technology and economics, but of the soul.
스물 한가지의 교훈은 철학, 기술, 경제학의 관점을 넘어 영혼에 이르는 진정한 비트코인 여행을 경험하게 해준다.

%As you dive deeper into the philosophy tersely laid out in 7 of the 21 Lessons,
%one can go as far as to understand the origin of all beings with enough time and
%contemplation. His 7 lessons on economics captures, in simple terms, how we are
%at the financial mercy of a small group of Mad Hatters, and how they have
%successfully managed to put blinders on our minds, hearts and souls. The 7
%lessons on technology lay out the beauty and technologically-Darwinian
%perfection of Bitcoin. Being a non-technical Bitcoiner, the lessons provide a
%salient review of the underlying technological nature of Bitcoin, and indeed,
%the nature of technology itself.

철학에 깊이 빠져들 수 있는 일곱 개의 교훈을 통해 존재의 기원을 이해할 수 있다. 
경제학에 대한 일곱 가지 교훈은 어떻게 기득권 세력들이 우리의 눈을 가릴 수 있었는지 알려준다. 
기술에 관한 일곱 가지 교훈은 비트코인의 미적으로 기술적으로 완벽함을 보여준다.
기술을 잘 모르는 비트코이너에게 비트코인의 기술적 특성과 기술의 필요성에 대해 이해할 수 있게 해준다.

%In this transient experience we call life, we live, love and learn. But what is
%life but a timestamped order of events?

이 일시적인 경험에서 우리는 살고 사랑하고 배운다. 
그러나 인생은 무엇인가? 이벤트의 타임스탬프가 아닐까?

%Conquering the Bitcoin mountain is not easy. False summits are rife, rocks are
%rough, and cracks and crevices are ubiquitously lying in wait to swallow you up.
%After reading this book, you will see that Gigi is the ultimate Bitcoin Sherpa,
%and I will appreciate him forever.

비트코인 산을 정복하는 것은 쉽지 않다. 
거짓된 정상은 만연하고 바위는 거칠고, 갈라진 틈은 당신을 집어삼키려고 한다.
이 책을 읽고 나면 Gigi가 궁극적인 비트코인 인도자임을 알게 될 것이며, 
나는 그를 영원히 높이 평가할 것이다.


\begin{flushright}
  %Hass McCook \\
  %November 29, 2019
  2019년 11월 29일 \\
  하스 맥쿡
\end{flushright}
