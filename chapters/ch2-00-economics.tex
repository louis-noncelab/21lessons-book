\part{경제학}
\label{ch:economics}
\chapter*{경제학}

\begin{chapquote}{루이스 캐롤, \textit{이상한 나라의 앨리스}}
\enquote{정원에 큰 장미나무가 서 있었는데, 그 장미는 흰색이었지만, 세 명의 정원사가 빠르게 빨간색으로 색칠했다.
앨리스는 이를 참 이상한 일이라고 생각했다\ldots}
\end{chapquote}

%Money doesn’t grow on trees. To believe that it does is foolish, and our
%parents make sure that we know about that by repeating this saying like a
%mantra. We are encouraged to use money wisely, to not spend it frivolously,
%and to save it in good times to help us through the bad. Money, after all,
%does not grow on trees.
돈은 나무에서 자라지 않는다. 
그렇게 믿는 것은 어리석은 일이며, 부모님은 이 말을 되풀이하며 가르친다.
우리는 돈을 현명하게 사용하고, 낭비하지 않아야 하며, 위기를 이길 수 있도록 저축하라고 말한다. 
돈은 나무에서 자라지 않는다.

%Bitcoin taught me more about money than I ever thought I would need to know.
%Through it, I was forced to explore the history of money, banking, various
%schools of economic thought, and many other things. The quest to understand
%Bitcoin lead me down a plethora of paths, some of which I try to explore in
%this chapter.
비트코인은 돈에 대해 알아야 할 것보다 더 많은 것을 알려주었다. 
나는 비트코인을 통해 돈의 역사, 은행, 경제 사상과 기타 많은 것들을 알 수 있었다. 
비트코인을 이해하기 위한 공부는 나에게 많은 깨우침을 주었으며, 그중 일부를 이 챕터에서 다루고자 한다.

%In the first seven lessons some of the philosophical questions Bitcoin touches
%on were discussed. The next seven lessons will take a closer look at money and
%economics.
다음 일곱개의 교훈은 돈과 경제에 대한 이야기이다.

~

\begin{samepage}
Part~\ref{ch:economics} -- Economics:

\begin{enumerate}
  \setcounter{enumi}{7}
  \item 금융 무지
  \item 인플레이션
  \item 가치
  \item 돈
  \item 역사와 돈의 몰락
  \item 부분 준비금의 광기
  \item 건전화폐
\end{enumerate}
\end{samepage}

%Again, I will only be able to scratch the surface. Bitcoin is not only
%ambitious, but also broad and deep in scope, making it impossible to cover all
%relevant topics in a single lesson, essay, article, or book. I doubt if it is
%even possible at all.
다시 강조하지만 표면적인 내용만 다룬다. 
비트코인은 범위가 넓고 깊어서 단일 강의, 에세이, 기사나 책에서 모든 주제를 다루는 것은 불가능하다. 


%Bitcoin is a new form of money, which makes learning about
%economics paramount to understanding it. Dealing with the nature of human action
%and the interactions of economic agents, economics is probably one of the
%largest and fuzziest pieces of the Bitcoin puzzle.
비트코인은 새로운 형태의 돈으로 비트코인을 이해하기 위해서는 무엇보다 경제에 대해 배우는 것이 중요하다.
인간 행동의 본질과 경제 주체의 행동을 다루는 경제학은 비트코인 퍼즐의 가장 큰 조각 중 하나이기 때문이다.

%Again, these lessons are an exploration of the various things I have learned
%from Bitcoin. They are a personal reflection of my journey down the rabbit hole.
%Having no background in economics, I am definitely out of my comfort zone and
%especially aware that any understanding I might have is incomplete. I will do my
%best to outline what I have learned, even at the risk of making a fool out of
%myself. After all, I am still trying to answer the question:
다시 말하지만, 이 교훈은 내가 비트코인에서 배운 다양한 것들을 설명하는 것이다. 
비트코인 토끼 굴을 여행하면서 얻은 나의 개인적인 반성이다. 
나는 경제학에 대한 배경 지식이 없기 때문에 나의 이해가 불완전할 수 있다. 
내가 다소 어리석을지 몰라도 내가 배운 것들을 최선을 다해 요약하였다. 
나는 아직 다음 질문에 대해 해답을 찾고 있다.\enquote{비트코인으로부터 무엇을 배웠는가?}

%After seven lessons examined through the lens of philosophy, let’s use the lens
%of economics to look at seven more. Economy class is all I can offer this time.
%Final destination: \textit{sound money}.
앞에서는 철학적 관점에서 일곱 가지의 교훈을 설명하였다.
이번에는 경제학적 관점을 사용하여 일곱 가지 교훈을 설명하고자 한다. 
나는 이 경제학에 대한 내용을 다루기 위해 최선을 다했다.
우리의 마지막 목적지는 건전화폐(sound money)이다.

% [the question]: https://twitter.com/arjunblj/status/1050073234719293440
